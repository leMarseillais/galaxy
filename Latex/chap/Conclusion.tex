\chapter*{Conclusion}
\addcontentsline{toc}{chapter}{Conclusion}

Les logiciels de Workflow tel Galaxy s’inscrivent dans une logique de pérennisation des processus analytiques, qui a pour but de sortir de la logique « projet » (dans le sens ponctuel) en créant des processus d’analyses génériques. Comme nous l'avons vu, un logiciel de Workflow est un outil permettant d’exécuter un ensemble de processus de façon automatique. Ces « pipelines » sont très présents en bioinformatique (à défaut d’être tres utilisés) car ils permettent aux chercheurs en biologie d’analyser leurs données (issues de séquencages, génotypages) de façon relativement transparente et (quasiment) sans l’aide d’informaticiens (denrées rares dans la recherche).

Toutefois, il conviendra de distinguer deux sortes de logiciel de Workflow :
\begin{itemize}
\item{Les logiciels de Workflow qui permettent aux chercheurs de manipuler leurs données et exécuter leurs analyses sans posséder de connaissances en écriture de scripts ou en bases de données. Les données sont rapatriées au sein du logiciel de Workflow, permettant l’exécution d’un ensemble de tâches, à travers des modules pré-installés. En séquençage, le Workflow permet de convertir des séquences en formats divers, les filtrer  ou les assembler .Le logiciel de Workflow ISYS (2001), BioMOBY, Taverna et Galaxy entrent dans cette catégorie.}

\item{Les logiciels de Workflow qui assurent un accès direct à des composants (installés sur le serveur) et/ou aux données génomiques sans passer par un rapatriement préalable des données. WildFire, Pegasys ou Ergatis (ce dernier sera décrit dans un prochain post) font partie de cette catégorie. De manière générale ces logiciels de Workflow sont plus difficiles à prendre en main mais sont évidemment plus flexibles.}
\end{itemize}
Pour résumer, Galaxy permet :
\begin{itemize}
\item{D’automatiser des processus d’analyse (idéalement répétitifs) en les reliant dans un pipeline}
\item{De lancer des analyses sur des architectures matérielles complexes telles des grilles de calculs ou des serveurs}
\item{ De formaliser le processus d’analyse en vue d’une publication scientifique}
\end{itemize}