\chapter{Présentation}

\section{État de l'art}

Nous présentons ici une revue des principaux logiciels "concurrents" ou du moins répondant aux mêmes types de problématiques que Galaxy. Certains parmi eux ne réalisent qu'une petite partie des traitements accomplis par Galaxy. Quant aux autres, rares sont ceux qui offrent autant de liberté et de facilité d'utilisation. De plus, la plupart sont payants et offre donc seulement des versions d'essais limités.

\subsection{CLC bio genomics workbench }
\subsubsection{Description}
C'est une plate forme commerciale et une extension du logiciel CLCbio Main Workbench .
On peut uniquement obtenir une version d'essai pour 15 jours.
\begin{itemize}
\item{assembleur de séquences de novo (inclus la détection des sNP , CHiP-seq).}
\item{Composition du génome.}
\item{Analyse des séquences (gènes et régions intergéniques).}
\item{Nombre de gènes et leur position sur les chromosomes.}
\item{Niveau d’expression des gènes.}
\item{Comparaison des génomes de diverses espèces (génomique comparative).}
\end{itemize}
\subsubsection{Système d'exploitation}
 Windows, Mac OS X et Linux.
\subsection{Genomatix}
\subsubsection{Description}
Genomatix propose des solutions et des services pour l'ensemble du déroulement de l'analyse, de la cartographie de premier niveau à l'intégration des résultats multiples avec des copies des fond des données de haute qualité. Les technologies de visualisation et d'interprétation permettent aux scientifiques du monde entier de transformer leurs données en résultats significatifs, transformant le séquençage nouvelle génération en un outil parfait appliqué à la médecine personnalisée. Il se décompose en trois grandes entités que sont Genomatix Genome Analyzer (GGA), Genomatix Station  mining (GMS) et Genomatix Software Suite.
\newpage
\subsubsection{Genomatix Genome Analyzer (GGA)}
Genomatix Genome Analyzer (GGA) est une solution intégrée complète pour la visualisation et l'interprétation de Next Generation Sequencing (NGS) de données de la puce à ARN, d'ADN,ou de petit séquençage d'ARN. Chaque analyseur est pourvu d'un état de l'art des technologies ce qui éclaire le contexte biologique. Le GGA produit des résultats de grande pertinence.\\
Les données de base biologiques comprenant des données du réseau d'annotation et de gènes fournis par Eldorado, ainsi que la connaissance des facteurs de transcription contenues dans MatBase permettent aux chercheurs d'analyser et d'interpréter leurs résultats expérimentaux dans un contexte biologique unique sur chaque GGA et sur plus de 30 espèces différentes. L'analyse de l'expression différentielle (jusqu'au niveau de transcription), la vérification du réseau, l' analyse de la littérature ne sont que quelques-unes des tâches qui peuvent être effectuées.
\subsubsection{Genomatix Station mining (GMS)}
Genomatix Station Mining (GMS) offre un alignement haute performance des séquençages de nouvelle génération (NGS). 
Elle permet de lire sur les génomes, transcriptomes, petit ARN. Avec son interface utilisateur intuitive, le GMS vous aide à exécuter rapidement des tâches telles que le positionnement génomique, SNP et détection d'Indel, des analyses d'épissage, la fusion de gènes et l'analyse structurelle. De plus, la station reste souple tout au long de l'expérience. En combinaison avec le Genome Analyzer Genomatix on obtient une solution d'analyse entièrement intégrée à partir du séquençage et pratique pour l'interprétation.
\subsubsection{Genomatix Software Suite}
Un bundle de logiciels bien établis, la suite logicielle Genomatix effectue un certain nombre de tâches:
\begin{itemize}
\item{elle procède à une analyse scientifique des données génomiques, la régulation des gènes et l'expression,}
\item{elle génère et évalue les réseaux et les voies,}
\item{elle effectue des recherches documentaires étendues et des analyses de séquence et de l'extraction,}
\item{elle visualise les annotations des génomes complets.}
\end{itemize}

\subsection{Ergatis}
\subsubsection{Description}

Ergatis est un utilitaire-Web  utilisé pour créer, exécuter et contrôler les pipelines réutilisables en analyse informatique. Il contient des composants pré-construits pour les tâches courantes d'analyse bioinformatique. Ces composants peuvent être disposées graphiquement pour former des pipelines hautement configurables. Chaque composant d'analyse soutient plusieurs formats de sortie, y compris le Systems Bioinformatic Markup Language (SBML) qui est un format ouvert XML pour l'échange de séquences et de leurs méta-données. L'implémentation actuelle inclut le support pour le chargement des données dans des bases de projet suivant le schéma Chado (le design d'une base de données particulière), un schéma normalisé soutenue par la communauté  pour le stockage des données biologiques.

Ergatis utilise le moteur de workflow pour traiter ses travaux sur une grille de calcul. Workflow fournit un langage XML et du moteur de traitement pour préciser les étapes d'un pipeline de calcul. Il fournit l'état d'exécution détaillé, facilite la récupération d'erreur au point de défaillance, et est hautement évolutive avec un support pour les environnements informatiques les plus distribués. Le format XML utilisé permet d'exécuter les commandes  en série, en parallèle, et dans n'importe quelle combinaison ou niveau d'imbrication.

Ce cadre de travail a été employé dans l'annotation de plusieurs grands organismes eucaryotes, y compris \textit{Aedes aegypti} et \textit{Trichomonas vaginalis}.
Ce projet est à ce jour fonctionnel tout en étant en cours de développement actif, avec la plupart des codes proviennent de contribution avec l'Institut des sciences du génome et le J. Craig Venter Institute.

\subsection{Biomoby}
\subsubsection{Description}

BioMoby est un projet Open Source de recherche qui vise à générer une architecture pour la découverte et la distribution des données biologiques à travers des services Web. Les données et les services sont décentralisés, mais la disponibilité de ces ressources, et les instructions permettant d'interagir avec eux, sont inscrits dans un emplacement central appelé MOBY centrale.\\
L'approche actuelle étend les services web en mettant en œuvre un modèle de registre novateur qui permet la recherche et la récupération basée sur l'objet et les hiérarchies de service. Cela permet aux utilisateurs de parcourir des données expansives et disparates où chaque étape possible est présenté sur la base des données de l'objet actuellement en main. Les Objets BioMoby natifs sont de type XML, et constituent à la fois la requête et la réponse d'un Simple Object Access Protocol (SOAP) de transaction.



\subsection{Lasergene 9.1}
\subsubsection{Description}
Lasergene 9.1 comprend de nouveaux assembleurs de séquences "next-gen", des workflows d'analyse et les séquences, structures et vues d'analyse des protéines intégrées. Cela fait  vraiment de Lasergene un des logiciels  intégrés les plus fiables du marché soutenant l'analyse de séquence traditionnelle, l'assemblage de séquences next-gen ainsi que leurs analyses, les études d'expression génique, l'analyse d'ARN-Seq et de ChIP-Seq.
\subsubsection{Système d'exploitation}
Windows, Mac et Linux
\subsection{JMP Genomics}
\subsubsection{Description}
JMP Genomics est un logiciel de découverte statistique issu de deux références dans les logiciels analytiques: SAS et JMP. Les organismes de recherche utilisent JMP Genomics pour découvrir des modes significatifs dans la génétique à haut débit, l'expression des génes, le nombre de copies et les données protéomiques. L'Interaction graphique dynamique rend facile à explorer les données en relation avec un ensemble complet d' algorithmes avancés de statistiques.

\subsubsection{Système d'exploitation}
On trouve une version 9 qui est compatible avec Windows et Mac. Les versions antérieures sont quand à elles compatibles aussi avec Linux.