\chapter*{Introduction}
\addcontentsline{toc}{chapter}{Introduction}

Le séquençage du génome, c'est à dire l'ensemble du code ADN d'un être vivant, s'est transformé ces dernières années en une course effrénée entre laboratoires concurrents.\\

Cette course a contribué à l'explosion de la quantité de données à traiter, données fournies par les séquenceurs nouvelle génération. L'étude de ces données nécessite de posséder des outils de grande fiabilité mais surtout possédant un large panel de traitements applicables à ses mêmes données.\\

 En effet, étudier le génome ne se limite pas à un simple déchiffrage du code contenu dans l'ADN, l'objectif reste principalement de comprendre le génome en analysant les gènes et les régions intergéniques, en évaluant le niveau d'expression de gènes par rapport à d'autres, en comparant les génomes de différentes espèces entre eux afin d'en sortir des lois et des règles.\\
 
 Il existe de nombreuses plates formes sur le web ainsi que de nombreux logiciels dédiés à l'étude des données issues du séquençage. Notre regard s'est porté sur Galaxy\footnote{\cite{afgan2010galaxy}\cite{afgan2011harnessing}\cite{afganreference}\cite{afgan2011galaxy}\cite{blankenberg2011integrating}\cite{blankenberg2010manipulation}\cite{blankenberg2011making}\cite{blankenberg2007framework}\cite{bock2010web}\cite{giardine2005galaxy}\cite{goecks2011galaxy}\cite{lazarus2008toward}\cite{miller200728}\cite{pond2009windshield}\cite{schatz2010missing}}, à la fois une plate forme web et un logiciel open source, qui présente une multitude de traitements à appliquer aux données du génome.\\
 
 Nous présenterons ainsi une revue sommaire des logiciels existants à ce jour puis nous vous exposerons une série de traitements réalisés afin de mettre en exergue les qualités et les défauts inhérents à ce logiciel.
